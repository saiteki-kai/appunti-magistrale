\chapter{Causal Models}
In un \textbf{randomized control experiment} tutti i fattori $X_1, ..., X_n$ che influenzano l'outcome $Y$ sono statici o variati randomicamente tranne uno  $X_i$.
In questo modo un cambiamento nell'outcome $Y$ sarà dovuto solo da quel fattore $X_i$.

Spesso non è possibile effettuare gli esperimenti per cause pratiche o etiche, quindi si eseguono degli studi osservazionali in un cui
si raccolgono i dati rispetto ad alterarli.

Il problema degli studi osservazionali è distinguere la causazione dalla correlazione.

\section{Intervention and do-Operator}
\begin{center}
  \begin{minipage}{0.475\linewidth}
    \centering
    \textbf{Intervening}

    Cambiamo il sistema fissando $X=x$ e osserviamo i cambiamenti su tutta la popolazione.
  \end{minipage}
  \begin{minipage}{0.475\linewidth}
    \centering
    \textbf{Conditioning}

    Non cambiamo nulla e consideriamo un sottoinsieme della popolazione $X=x$.
  \end{minipage}
\end{center}

Denotiamo l'intervention con il do-operator $do(X=x)$.
\begin{align*}
  P(Y(x) = y) = P(Y = y | do(X = x)) = P(y | do(x))
\end{align*}
%
\begin{align*}
   & P(Y | X = x) = P(y | x)         \qquad & \text{observational distribution}  \\
   & P(Y | do(X = x)) = P(y | do(x)) \qquad & \text{interventional distribution}
\end{align*}

Un'espressione con $do$ è detta interventional expression, mentre senza $do$ è detta observational expression.

Un'interventional expression che può essere ridotta a una observational expression è detta \textbf{identificabile}.

Una stima è detta causale se contiene il do-operator, statistica altrimenti.

\bigskip
\begin{center}
  \begin{minipage}{0.475\linewidth}
    \centering
    \textbf{pre-intervention distribution}
    \begin{align*}
      P(Y | do(x), Z = z)
    \end{align*}
  \end{minipage}
  \begin{minipage}{0.475\linewidth}
    \centering
    \textbf{post-intervention distribution}
    \begin{align*}
      P(Y | x, Z = z)
    \end{align*}
  \end{minipage}
\end{center}
\bigskip

\section{Modularity and Adjustment Formula}
Il \textbf{causal mechanism} è un meccanismo che genera $X_i$ come distribuzione condizionale di $X_i$ dati i genitori (cause) $pa(X_i)$, ovvero $P(X_i | pa(X_i))$.

Assunzione: Le intervention sono locali

Un intervention su una variabile $X_i$ cambia solo il suo causal mechanism, non cambia il causal mechanism che genera un'altra variabile $X_j$.

\subsection*{Modularity - Independence Mechanism - Invariance}
Se interveniamo su un insieme di variabili $S$ fissando valori costanti,
allora per ogni $X_i \in \{X_1, ..., X_n\}$
\begin{enumerate}
  \item Se $X_i \in S$, allora $P(X_i = x| pa(X_i)) = 1$ se $x$ è il valore assegnato dall'intervention $do(X_i=x)$, 0 altrimenti
  \item Se $X_i \not\in S$, allora $P(X_i = x | pa(X_i))$ rimane inalterato
\end{enumerate}

Data una variabile $X_i \in S$, un valore $x$ di $X_i$ è \textbf{consistente} con l'intervention su $X_i$ se $x$ è uguale al valore che è stato assegnato nell'intervention $do(X_i=x)$.

Una volta impostato il valore di $X_i$ non importa più il valore dei nodi genitori $pa(X_i)$, poiché al variare di $pa(X_i)$ non varierà il valore di $X_i$.
Questo possiamo rappresentarlo attraverso un grafo (manipolato) in cui si rimuovono gli archi entranti, se non ce ne sono l'intervention non ha conseguenze sulla distribuzione post-intervention
$p(y|do(x)) = p(y|x)$.

\subsection*{Adjustment Formula}
\# TODO

\subsection*{Causal Effect Rule}
Generalizzazione della adjustment formula.

Dato un grafo $G$ in cui un insieme di variabili $pa(X)$ sono i genitori di $X$
\begin{align*}
  P(Y=y | do(X=x)) & = \sum_u {P(Y=y | X=x, pa(X)=u) P(pa(X)=u)}              \\
                   & = \sum_u {\frac{P(Y=y, X=x, pa(X)=u)}{P(X=x | pa(X)=u)}}
\end{align*}

\section{Backdoor Adjustment}
