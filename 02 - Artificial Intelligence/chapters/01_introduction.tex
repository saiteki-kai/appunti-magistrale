\chapter{Introduction}

\section{Acting Humanly (Turing Test Approach)}
Alan Turing ha progettato il \textbf{Turing Test} 
per fornire una definizione operazionale di intelligenza.

Per passare il test un computer deve possedere le seguenti capacità:
\begin{itemize}
  \item NLP: comunicare in una lingua comprensibile all'uomo
  \item Knowledge Representation: memorizzare quello che sa e che ascolta
  \item Automated Reasoning: usare le informazioni raccolte per formulare risposte e trarre conclusioni
  \item Machine Learning: adattarsi ed estrapolare pattern
\end{itemize}
In aggiunta per avere anche una simulazione fisica di intelligenza sono necessari:
\begin{itemize}
  \item Computer Vision: percepire oggetti
  \item Robotics: manipolare oggetti e muoversi
\end{itemize}

\section{Thinking Humanly (Cognitive Approach)}
Per poter dire che un computer pensa come un umano è necessario prima determinare il funzionamento della mente umana.
Da questo è nato il campo della \textbf{cognitive science} che unisce i modelli di intelligenza artificiale agli esperimenti psicologici
per costruire e testare teorie su come pensa un umano. 

\section{Acting Rationally ("laws of thought" Approach)}
Questo approccio nasce dall'idea che delle leggi di pensiero (laws of thought) governino la mente umana.
Si è cercato quindi di tradurre la conoscenza in linguaggio logico con l'obiettivo di costruire sistemi intelligenti
basati sulla logica.

\section{Thinking Rationally}
Questo approccio si basa sul concetto di agente, ovvero qualsiasi cosa che agisce. 
Un \textbf{agente razionale} è un agente che agisce per raggiungere il miglior risultato.
