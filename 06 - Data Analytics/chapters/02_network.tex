\chapter{Networks Analytics}

\section{Statistiche Descrittive}

\subsection*{Grado}
Il \textbf{grado} $k$ di un nodo $i$ è il numero di archi entranti e uscenti.
E' possibile distinguere il grado in \textbf{outdegree} $k_i^{\text{out}}$ e \textbf{indegree} $k_i^{\text{in}}$, il grado totale è dato dalla somma $k = k_i^{\text{in}} + k_i^{\text{out}}$.

In un grafo \underline{diretto} si definisce \textbf{source} un nodo con $k_i^{\text{in}} = 0$, e \textbf{sink} $k_i^{\text{out}} = 0$.

\subsubsection*{Grado Medio}

Grafi non diretto:
\begin{align*}
   & \left\langle k \right\rangle = \frac{1}{N} \sum_{i=1}^N k_i &
   & \left\langle k \right\rangle = \frac{2|E|}{N}               &
\end{align*}

Grafi diretto:
\begin{align*}
   & \left\langle k^{in} \right\rangle = \frac{1}{N} \sum_{i=1}^N k_i^{in}   &
   & \left\langle k^{out} \right\rangle = \frac{1}{N} \sum_{i=1}^N k_i^{out} &
\end{align*}
\begin{align*}
   & |E| = \left\langle k^{in} \right\rangle = \left\langle k^{out} \right\rangle &
   & \left\langle k \right\rangle = \frac{|E|}{N}                                 &
\end{align*}

\subsubsection*{Distribuzione del Grado}
Si definisce $P(k)$ la probabilità di un nodo di avere grado $k$ e rappresenta la \textbf{distribuzione del grado} $P(k) = \text{N}_k / \text{N}$, dove $\text{N}_k$ é il numero di nodi con grado $k$.

\subsection*{Distanza e Diametro}
La \textbf{distanza} tra due nodi A e B è definita come il numero di archi lungo lo \textit{shortest path} da A a B.
Se i due nodi non sono \textbf{raggiungibili} tra loro la distanza é infinito.

In un grafo \underline{diretto} la distanza tra due nodi non é simmetrica, $\text{d}(\text{A}, \text{B}) \not= \text{d}(\text{B}, \text{A})$.

Per calcolare le distanze in un grafo si utilizza la \textbf{breadth-first search (BFS)}.

Il \textbf{diametro} di un grafo é la distanza massima tra ogni coppia di nodi nel grafo.

La \textbf{distanza media} in un grafo si definisce:

\begin{minipage}{0.45\textwidth}
  \centering

  \begin{align*}
    \left\langle d \right\rangle = \frac{1}{2|E_{max}|} \sum_{i, j\not=i} d_{ij}
  \end{align*}
  \small\textit{grafo diretto}
\end{minipage}
\hfill
\begin{minipage}{0.45\textwidth}
  \centering

  \begin{align*}
    \left\langle d \right\rangle = \frac{1}{|E_{max}|} \sum_{i, j>i} d_{ij}
  \end{align*}
  \small\textit{grafo indiretto}
\end{minipage}

\subsection*{Coefficiente di Clustering}
Il \textbf{coefficiente di clustering} é una misura di quanto i nodi vicini tendono a collegarsi tra loro.
Dato un nodo $i$ con grado $k_i$ il coefficiente di clustering é definito come:

\begin{minipage}{0.45\textwidth}
  \centering

  \begin{align*}
    C_i = \frac{2|E_i|}{k_i (k_i - 1)}
  \end{align*}
  \small\textit{grafo diretto}
\end{minipage}
\hfill
\begin{minipage}{0.45\textwidth}
  \centering
  \begin{align*}
    C_i = \frac{|E_i|}{k_i (k_i - 1)}
  \end{align*}
  \small\textit{grafo indiretto}
\end{minipage}

in cui l'insieme $E_i = \{e_{jk} : v_j, v_k \in N_i, e_{jk} \in E\}$ contiene tutti gli archi che sono collegati al nodo $i$ e $N_i = \{v_j : e_{ij} \in E \lor e_{ji} \in E \}$ é il suo \textbf{vicinato}, ovvero i nodi collegati direttamente a $i$.

Il coefficiente di clustering assume un valore tra 0 e 1:
\begin{itemize}
  \item Se $C_i = 0$, allora nessun nodo del vicinato sarà collegato agli altri.
  \item Se $C_i = 1$, allora tutti nodi del vicinato saranno collegati tra loro.
\end{itemize}

E' possibile catturare il grado di clustering di un grafo con il \textbf{coefficiente medio di clustering}, ovvero la probabilità che dato un nodo qualsiasi due nodi nel suo vicinato siano collegati tra loro.
\begin{align*}
  \left\langle C \right\rangle = \frac{1}{N} \sum_{i=1}^N C_i
\end{align*}

\subsection*{Centralizzazione}
