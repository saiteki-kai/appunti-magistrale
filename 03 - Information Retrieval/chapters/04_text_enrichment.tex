\chapter{Text Enrichment}

Il text enrichment permette di avere informazioni aggiuntive sul testo che possono aiutare a
fare delle analisi più approfondite e predizioni più precise.

Due tecniche che ci permetto di arricchire il testo sono il POS tagging e la Named Entity Recognition.
Entrambe le tecniche possono essere applicate usando degli insieme di regole o degli approcci di apprendimento supervisionato.

\section{Part-of-Speech (POS) tagging}
Il POS tagging è il processo che marca ogni termine nel documento con un tag che corrisponde a una part-of-speech.
%
{\small\begin{align*}
  \frac{\text{A}}{\color{orange}{\textbf{DET}}} +
  \frac{\text{dog}}{\color{teal}{\textbf{NOUN}}} +
  \frac{\text{is}}{\color{red}{\textbf{AUX}}} +
  \frac{\text{chasing}}{\color{blue}{\textbf{VERB}}} +
  \frac{\text{a}}{\color{orange}{\textbf{DET}}} +
  \frac{\text{boy}}{\color{teal}{\textbf{NOUN}}} +
  \frac{\text{in}}{\color{violet}{\textbf{PREP}}} +
  \frac{\text{the}}{\color{orange}{\textbf{DET}}} +
  \frac{\text{park}}{\color{teal}{\textbf{NOUN}}}
\end{align*}}%

\subsection*{Word Classes}
Le classi di parole hanno le seguenti caratteristiche:
\begin{itemize}
  \item hanno trasformazioni simili
  \item hanno le stesse funzioni nella frase
  \item appaiono in un contesti simili
\end{itemize}
Le word class possono essere chiamate anche part-of-speech, lexical categories, morphological classes, lexical tags o POS.

Per effettuare il POS tagging è necessario scegliere un insieme di tag da associare alle parole.
Comunemente vengono identificate 9 classi di POS: noun, verb, adjective, adverb, preposition, article, interjection e conjunction.
Comunque è possibile usare più classi e sottoclassi.

\subsection*{Tag Ambiguity}
Spesso una parola può essere associata a più di un POS, quindi è necessario considerare il contesto.

\begin{quote}
  \bigskip
  \centering
  \begin{minipage}{0.8\linewidth}
    \begin{itemize}
      \item The \textbf{back} door \hfill{(adjective)}
      \item Promised to \textbf{back} the bill \hfill{(verb)}
      \item Win the voters \textbf{back} \hfill (adverb)
      \item On my \textbf{back} \hfill (singular noun)
    \end{itemize}
  \end{minipage}
  \bigskip
\end{quote}

E' possibile disambiguare le parole considerando il contesto. Spesso dei tag co-occorrono con altri tag (e.g. gli articoli e i sostantivi).

\subsection*{Rule Based Tagging}
Si assegna ad ogni parola tutti i possibili tag usando un dizionario.
Tramite delle regole scritte a mano, vengono rimossi alcuni tag finché non ne rimane uno solo.

\section{Named Entity Recognition (NER)}
Il Named Entity Recognition (NER) consiste nel trovare e classificare entità in un testo (persone, date, luoghi, organizzazioni).

Il NER necessita di definire delle categorie di interesse.
Tre categorie universalmente accettate sono: persone, luoghi e organizzazioni.
Altri task possono richiedere il riconoscimento di data/ora, espressioni, misure e in alcuni casi
è utile usare entità specifiche del dominio (farmaci, navi, elementi chimici, ...).

IL NER risulta difficile nei casi di \textbf{metonimia}, ovvero la sostituzione di un termine con un'altro
che ha una certa relazione con la prima e mantiene in un certo modo il significato.

\begin{itemize}
  \item “mi piace leggere Dante” / le opere di Dante (scambia l'autore per l'opera)
  \item “confidare nell'amicizia” / negli amici (scambia l'astratto per il concreto)
  \item “ha una buona penna” / sa scrivere bene (scambia la causa per l'effetto)
\end{itemize}

\subsection*{Rule Based}
Si usa una lista di parole e frasi che categorizzano le entità e delle regole per verificare o trovare nuove entità.
Le regole e le categorie dipendono dal linguaggio.

\begin{itemize}
  \item “$<$number$>$ $<$word$>$ street” per gli indirizzi
  \item “$<$street address$>$, $<$city$>$” o “in $<$city$>$” per i nome di città
  \item “$<$title$>$ $<$name$>$” per trovare nuovi nomi
\end{itemize}

\subsection*{Statistical Machine Learning}
Le entità vengono trovate stimando la probabilità che appaia insieme ad altre parole.
(e.g. “marathon” è un luogo o evento sportivo, mentre “boston marathon” è uno specifico evento sportivo.)
Le probabilità vengono ottenute usando un training set etichettato.

Un possibile approccio sono le Hidden Markov Model. Questo modello si basa sulla proprietà Markoviana, per cui
la prossima parola in una sequenza dipende solo dalla precedente.
L'entità viene riconosciuta dalla sequenza di parole con la probabilità più alta.
