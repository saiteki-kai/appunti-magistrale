\chapter{Text Enrichment}

riconoscere una frase:
- n-grams
- pos tagger
- store words position in a index

POS, NER approaches:
- rules based
- supervised learning

\section{Part-of-Speech (POS) tagging}
Il POS tagging è il processo che marca ogni termine nel documento con un tag che corrisponde a una part-of-speech.

EXAMPLE...

\subsection*{Word Classes}
words that somehow behave alike:
\begin{itemize}
  \item similar transformations
  \item similar functions in the phrase
  \item similar contexts
\end{itemize}

9 traditional word classes of POS
(noun, verb, adjective, adverb, preposition, article, interjection, conjunction)

Applicazioni:
- Machine Translation
- Parsing
- Speech Recognition

\subsection*{Tag Ambiguity}
Spesso una parola può essere associata a più di un POS, quindi è necessario considerare il contesto.

\begin{quote}
  \bigskip
  \centering
  \begin{minipage}{0.8\linewidth}
    \begin{itemize}
      \item The \textbf{back} door \hfill{(adjective)}
      \item Promised to \textbf{back} the bill \hfill{(verb)}
      \item
    \end{itemize}
  \end{minipage}
  \bigskip
\end{quote}

\subsection*{Rule Based Tagging}
- assegno ogni possibile tag a una parole usando un dizionario
- scrivo delle regole a mano per rimuove dei tag
- lascio un solo tag per parola

consideriamo le frasi con n-grammi

\section{Named Entity Recognition (NER)}
Trovare e classificare nomi in un testo (persone, date, luoghi, organizzazioni).
