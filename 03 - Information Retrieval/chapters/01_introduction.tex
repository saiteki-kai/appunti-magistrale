\chapter{Definitions}

\section{Document}
Un \textbf{documento} è solitamente formato da un testo, una struttura, altri media (immagini, suoni, ...) e da dei metadata.

Per \textbf{testo} si intende una sequenza di stringhe di caratteri di un alfabeto.\\
E.g. le sequenze del genoma, formule chimiche e parole del linguaggio naturale. 

Un documento può essere composto da
\begin{itemize}
  \item structured data (tabelle, database, ...)
  \item semi-structured data (html, xml, ...)
\end{itemize}

I \textbf{metadati} sono dati esterni riguardo al documento. Possono essere classificati in due categorie:

\begin{itemize}
  \item \textbf{metadati descrittivi}: riguardano la creazione del documento (e.g. titolo, autore, data, ...)
  \item \textbf{metadati semantici}: descrivono informazioni contestualmente rilevanti o specifiche del dominio (e.g. ontologie)
\end{itemize}

\section{Terms}
I termini sono dei descrittori che vengono associati al testo.

\section{Stop Words}
I termini che non sono significativi per la rappresentazione del testo (particelle, articoli, ...).
