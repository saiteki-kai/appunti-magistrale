\chapter{Text Processing}
Il text processing è una fase necessaria per preparare e pulire il testo.

\section{Tokenization}
La tokenization consiste nell'identificare e separare all'interno di un testo delle unità chiamate token.
I token possono essere parole, frasi, simboli o n-grammi. Ogni token è un candidato a essere un termine significativo (index).

\begin{quote}
  e.g.\ ``Text mining is to identify useful information''

  \textbf{Tokens}:``Text'',``mining'',``is'',``to'',``identify'',``useful'',``information''
\end{quote}

Problemi:
\begin{itemize}
  \item parole composte (``Hewlett-Packard'' $\rightarrow$ ``Hewlett'', ``Packard'')
  \item numeri, date (``Mar. 12, 1991'', ``12/3/1991'', ``(800) 234-2333'')
  \item problemi linguistici (parole composte, assenza di spazi, ...)
\end{itemize}

I token possono essere raggruppati in sequenze contigue di N elementi chiamate N-grammi.

\begin{quote}
  e.g.\ ``Corpus is the collection of text documents.''

  \textbf{Bigrammi}:``Corpus is'', ``is the'', ``the collection'', ``collection of'', \\``of text'', ``text documents'', ``documents .''
\end{quote}

La tokenization si può effettuare tramite espressioni regolari o metodi statistici.

\newpage

\section{Normalization}
% TODO: fix
Ad un parola possono essere associati diversi token. La normalizzazione consiste nell'ottenere le classi di equivalenza dei token rimuovendo punti, trattini, accenti.

\begin{quote}
  U.S.A. $\Leftrightarrow$ USA\\
  anti-aliasing $\Leftrightarrow$ antialiasing\\
  résumé $\Leftrightarrow$ resume\\
  15/10/2021 $\Leftrightarrow$ 15 Ott 2021
\end{quote}

\subsection*{Lemmatization}
Le parole vengono ridotte alla loro forma base (lemma) tenendo in considerazione l'intero vocabolario della lingua e analizzando la parte del discorso.

\begin{quote}
  e.g.\ ``ladies'' $\Rightarrow$ ``lady'', ``forgotten'' $\Rightarrow$ ``forgot''
\end{quote}

\subsection*{Stemming}
Le parole vengono ridotte a una radice (stem) rimuovendo le flessioni tramite l'eliminazione dei caratteri non necessari.

\begin{quote}
  e.g.\ ``automate(s)'', ``automation'', ``automatic'' $\Rightarrow$ ``automat''
\end{quote}

\subsection*{Case folding}
Tutte le parole vengono convertite in lowercase a parte alcune eccezioni.

\subsection*{Thesaurus and Soundex}
% TODO: aggiungere motivo
Un thesaurus (tesauro) è una risorsa linguistica generata manualmente da essere umani in cui è possibile esprimere relazioni tra parole (e.g.\ gerarchie, sinonimi, ...).

Soundex è un algoritmo fonetico che permette di rappresentare correttamente diverse parole omofone nonostante differenze di ortografia usando delle euristiche fonetiche.


\section{Stop Words Removal}
Le \textbf{stop words} sono le parole più frequenti all'interno di un testo che possono essere rimosse senza perdere il significato.
Queste parole essendo presenti in più documenti non portano informazioni utili per distinguerli.

Esistono delle liste di \textbf{stop words} definite in base alla lingua che possono essere usate per la rimozione.

I web search engine non effettuano la rimozione delle \textbf{stop words} perché sono necessarie per alcune ricerche.
