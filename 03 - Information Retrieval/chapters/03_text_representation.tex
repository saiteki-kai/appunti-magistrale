\chapter{Text Representation}
% exploration analysis: (text clustering, ...) unsupervised learning beacause i don't have precedent knowledge
% Dopo aver fatto il processing sul testo si ottiene un insieme di discrettori, indici.

Un modo semplice per rappresentare un testo è una matrice in cui sulle righe ci sono termini estratti dal corpus (vocabolario) e sulle colonne i documenti.

\textbf{Incidence Matrix}: specifica la presenza di un termine in un ogni documento.

Ogni documento può essere rappresentato da un insieme di termini o da un vettore binario.

\begin{minipage}[c]{0.55\linewidth}
  \begin{tabular}{c >{\columncolor{yellow}}c c c c}
                                 & {\color{blue} \textbf{Doc1}} & {\color{blue} \textbf{Doc2}} & {\color{blue} \textbf{Doc3}} & {\color{blue} \textbf{Doc4}} \\
    {\color{red} \textbf{Term1}} & 1                            & 1                            & 1                            & 0                            \\
    {\color{red} \textbf{Term2}} & 0                            & 1                            & 1                            & 1                            \\
    {\color{red} \textbf{Term3}} & 0                            & 0                            & 1                            & 0
  \end{tabular}
\end{minipage}
\hfill
\begin{minipage}[c]{0.4\linewidth}
  $\text{Rappresentazione di Doc1}$
  \medskip

  $R1 = \{\text{Term1}, \text{Term2}, \text{Term3}\}$\\
  $R1 = \langle1, 0, 0\rangle$
\end{minipage}

\bigskip

\textbf{Count Matrix}: specifica la numero di occorrenze di un termine in ogni documento.

Un documento viene rappresentato da un vettore di occorrenze.

\begin{minipage}[c]{0.55\linewidth}
  \begin{tabular}{c >{\columncolor{yellow}}c c c c}
                                 & {\color{blue} \textbf{Doc1}} & {\color{blue} \textbf{Doc2}} & {\color{blue} \textbf{Doc3}} & {\color{blue} \textbf{Doc4}} \\
    {\color{red} \textbf{Term1}} & 57                           & 57                           & 71                           & 133                          \\
    {\color{red} \textbf{Term2}} & 4                            & 34                           & 17                           & 92                           \\
    {\color{red} \textbf{Term3}} & 232                          & 2                            & 10                           & 293
  \end{tabular}
\end{minipage}
\hfill
\begin{minipage}[c]{0.4\linewidth}
  $\text{Rappresentazione di Doc1}$
  \medskip

  $R1 = \langle157, 4, 232\rangle$
  \medskip
\end{minipage}

Le rappresentazioni vettoriali non considerano l'ordine delle parole nel testo.

\newpage

\section{Bag Of Words}
La Bag Of Words (BOW) è una rappresentazione del testo che descrive le occorenze di parole in un documento.

Bag of words con N-grammi\\
Pro: cattura le dipendenze locali e l'ordine\\
Contro: incrementa la frequenza delle parole

\subsection*{Zipf's Law}
Descrive la frequenza di un evento (parola) in un insieme in base al suo rank.

\textbf{rank}: posizione di un termine nell'ordine decrescente di frequenza dei termini in tutta la collezione.

La frequenza di una parola w, f(w) è proporzionale a 1/r(w).
%
\begin{flalign*}
  &f \propto \frac{1}{r} \quad\Rightarrow\quad f \cdot r = k \; \text{(costante)}&\\
  &P_r = \frac{f}{N} = \frac{A}{r} \qquad \text{probabilità del termine di rank r, } \quad A = \frac{k}{N} \approx 0.1 &
\end{flalign*}

\subsection*{Luhn's Analysis}
Generalmente termini con frequenza molto alta e molto bassa sono inutili per discriminare i documenti.
\\
\pgfmathdeclarefunction{gauss}{3}{%
  \pgfmathparse{1/(#3*sqrt(2*pi))*exp(-((#1-#2)^2)/(2*#3^2))}%
}

\begin{tikzpicture}
  \begin{axis}[
      no markers,
      domain=0:7,
      samples=150,
      ymin=0,
      ymax=1,
      axis lines*=left,
      height=8cm,
      width=10cm,
      ylabel=Frequency of words,
      xtick=\empty,
      ytick=\empty,
      enlargelimits=false,
      axis on top,
      clip=false
    ]

    \addplot[thick,dashed,cyan!50!black]
    {gauss(x, 3, 1)};
    \addplot[red!50!black,domain=0.2:7,name path=curve,restrict y to domain=-inf:1]
    {1/(4*x)};
    \addplot[name path=xaxis]
    {0};

    \addplot[orange!30]
    fill between[of=curve and xaxis,soft clip={domain=1:5}];

    \draw[gray]
    (axis cs:1,0) -- (axis cs:1,1)
    (axis cs:5,0) -- (axis cs:5,1);

    \node[right,align=left,anchor=north west] at (axis cs:1,1)  {Upper \\ cut-off};
    \node[right,align=left,anchor=north west] at (axis cs:5,1)  {Lower \\ cut-off};

    \coordinate (aux1) at (axis cs:3.5,{gauss(3.5,3,1)});
    \node[align=left,anchor=south west]
    at ([xshift=0.5cm,yshift=10pt]aux1)
    (respow)
    {Resolving power \\ of significant words};
    \draw
    (respow.west) -- (aux1);

    \coordinate (aux2) at (axis cs:3.5,{1/(4*3.5)});
    \node[align=left,anchor=south west]
    at ([xshift=0.3cm]aux2)
    (sig)
    {Significant words};
    \draw
    (sig.west) -- ([yshift=-4pt]aux2);

    \node[anchor=north west]
    at (axis cs:0,0)
    {Words by rank order};
  \end{axis}
\end{tikzpicture}


L'abilità delle parole di discriminare il contenuto di un documento è massimo nella posizione tra i due livelli di cut-off.

Vogliamo assegnare dei pesi ai termini.
\begin{itemize}
  \item corpus-wide: alcuni termini portano più informazione riguardo al documento
  \item document-wide: non tutti i termini sono ugualmente importanti  
\end{itemize}

TF (Term Frequency): within document\\
IDF (Inverse Document Frequency): whole collection

Il peso di un termine deve essere proporzionale a TF e inversamente proporzionale a IDF

$tf_{t,d}$: numero di occorenze del termine t nel documento d

$w_{t, d} = tf_{t,d} / max_{ti} tf_{ti,d}$

$df_t$ document frequency: numero di documenti che contiene t

$df_t \leq N$

definiamo la inverse document frequency (idf)

$idf_t = log(N/df_t)$

tf-idf weight\\
$w_{t, d} = tf_{t,d} / max_{ti} tf_{ti,d} * log(N/df_t)$
